\section{Zero Knowledge Proofs}

\subsection{Malicious Security}

A malicious adversary may not necessarily follow the protocol i.e. acts arbitrarily.

\putfigure{Malicious security}{malicious-security1}

\subsection{Commitment Schemes}

\begin{defn}
A \textbf{commitment} scheme satisfies two properties:
\begin{itemize}
    \item \textbf{Binding.} after the commitment phase, there should be at most on $m$ that $S^*$ could validly decommit to
    \item \textbf{Hiding.} after the commitment phase, $R^*$ has no information about $m$
\end{itemize}
\putfigure{Setup for commitment schemes}{commitment-setup}
\end{defn}

\subsection{Zero-Knowledge Proofs}

Let language $L \in \text{NP}$ i.e. there exists an efficient $R_L$ such that $x \in L \iff \exists w : R_L(x,w) = 1$.
Examples: SAT, HAM.
Note the following setup:

\putfigure{Setup for defining zero-knowledge proofs}{zk-setup}

\begin{defn}
A protocol for computing \textbf{zero-knowledge proofs} is one that evaluates $\mathcal{F}_{2k}$ against a malicious verifier $V$.
\end{defn}

\begin{defn}
A protocol for computing \textbf{zero-knowledge proofs of knowledge} (ZKPoK) is one that evaluates $\mathcal{F}_{2k}$ against a malicious prover $P$.
\end{defn}

We proceed by giving a ZKPoK protocol for an NP-complete language (HAM), which implies that there is a ZKPoK protocol for each language in NP.

\putfigure{A protocol for ZKPoK for HAM}{zkpok-ham1}

\begin{thm}
The above protocol is zero knowledge.
\end{thm}
\begin{proof}
TODO: lecture 10
\end{proof}

\begin{thm}
The above protocol is a proof of knowledge.
\end{thm}
\begin{proof}
TODO: lecture 10
\end{proof}

Problem: we know how to do a single ZKPoK, but we do not know how to prove that it is ZK in parallel repetition.

\begin{prcl}[KE]
The protocol $KE(G)$ runs as follows:
\begin{enumerate}
\item run an honest interaction with $P^*$; let challenge be $\vec{b}$
\item if interaction fails, halt
\item otherwise, set $\vec{b''} = \vec{0}$ and do:
\begin{enumerate}
    \item $\vec{b'} \gets \bits^n$
    \item if $P^*(\vec{b'})$ succeeds then $\vec{b'} \neq \vec{b}$, then break
    \item if $P^*(\vec{b''})$ succeeds then $\vec{b''} \neq \vec{b}$, then break
    \item if $\vec{b''} = \vec{1}$, break
    \item otherwise, increment $\vec{b''}$
\end{enumerate}
\item given two successful executions for distinct challenges, compute a witness $w$
\end{enumerate}
Let $\epsilon$ be the probability that $P^*$ succeeds.
\end{prcl}

\begin{thm}
If $\epsilon > 1/2^n$, then KE computes a witness with probability $\epsilon$. Also, KE runs in polynomial expected-time.
\end{thm}
\begin{proof}
TODO: lecture 11
\end{proof}

\begin{defn}
A PoK has \textbf{witness indistinguishability} (WI) if a cheating $V^*$ cannot distinguish which of two possible witnesses $P$ is using.
\putfigure{Witness indistinguishability}{witness-indistinguishability}
\end{defn}

Note that ZK $\implies$ WI.

\begin{prcl}[Goldeich-Kahan]
\putfigure{The Goldeich-Kahan protocol}{Goldeich-Kahan}
\end{prcl}

\begin{thm}
The Goldeich-Kahan protocol is ZK.
\end{thm}
\begin{proof}
TODO: lecture 11
\end{proof}

\begin{prcl}[Feige-Shamir]
Let $f$ be a one-way function.
\putfigure{The Feige-Shamir protocol}{Feige-Shamir}
\end{prcl}

\begin{thm}
The Feige-Shamir protocol is a ZKPoK.
\end{thm}
\begin{proof}
TODO: lecture 11
\end{proof}