\section{Security Definitions}

\subsection{Indistinguishable Distributions}

\begin{defn}
A function $f : \bN \to \bits$ is a \textbf{negligible function} when
\[
    \forall c \in \bZ:
    \exists N:
    \forall n \leq N:
    f(k) \leq \frac{1}{k^c}
\]
\end{defn}

\begin{defn}
A set $X = \set{X(k, a)}_{k in \bN, a \in \bits^*}$ is a \textbf{distribution ensemble}.
\end{defn}

\begin{defn}
Distributions $X$ and $Y$ are \textbf{perfectly indistinguishable} when 
\[
    \forall k, a:
    X(k, a) = Y(k, a)
\]
\end{defn}

\begin{defn}
Distributions $X$ and $Y$ are \textbf{statistically indistinguishable} with \textbf{$\epsilon$-closeness} when
\[
    \forall k, a:
    \std{X(k, a), Y(k, a)} \leq \epsilon(k)
\]
where 
\[
    \std(X, Y) =
    \frac{1}{2} \sum_a \abs{\prob{X = a} - \prob{Y = a}}
\]
and $\epsilon$ is a negligible function/
\end{defn}

\begin{defn}
Distributions $X$ and $Y$ are \textbf{computationally indistinguishable} if for all polytime distinguishers $D$, there exists a negligible function $\epsilon$ such that for all $a, z$:
\begin{align*}
    \abs{
        \prob{D(k, a, z, X(k, a))} = 1 -
        \prob{D(k, a, z, Y(k, a)) = 1}
    }
    \leq \epsilon(k)
\end{align*}
where $z$ is an auxiliary input.
\end{defn}

\subsection{Secure Computation}

Fix a randomized function $f$ from $n$ inputs to $n$ outputs.
How should we define security?
Important concepts:
\begin{itemize}
    \item real/ideal paradigm -- real-world execution should be ``close'' to ideal
    \item define real world
    \item define ideal world
    \item defined notion of ``closeness''
\end{itemize}

\begin{defn}
\textbf{Real-world} execution is defined by a protocol $\Pi$ with adversary $A$.
A \textbf{passive} adversary follows the protocol.
An \textbf{active} adversary behaves arbitrarily.

\[
\Real_{\Pi, A}(k, \vec{x}, z)
\]
\end{defn}

\begin{defn}
\textbf{Ideal-world} execution of $f$ takes into account:
\begin{itemize}
    \item item substitution
    \item computation
    \item output
    \item aborting?
\end{itemize}

\[
\Ideal_{\Pi, A}(k, \vec{x}, z)
\]
\end{defn}

\begin{defn}
A protocol $\Pi$ is \textbf{$t$-secure} if for all probabilistic polytime (PPT) adversaries $A$ corrupting at most $t$ parties, there exists a polytime $S$ corrupting at most $t$ parties such that 
\[ 
    \set{(\View_{\Pi, A}(k, \vec{x}, z), \Out_{\Pi, A}^H(k, \vec{x}, z))}
    \approx 
    \set{(\Out_{f,S}^S(k, \vec{x}, z), \Out_{f,S}^H(k, \vec{x}, z))}
\]
are computationally indistinguishable.
\end{defn}

\subsection{Hybrid World}

\begin{defn}
A \textbf{hybrid-world} protocol $\Pi$ evaluating $f$ is secure if for all PPT $A$, there exists a PPT $S$ such that
\[
    \Hybrid_{\Pi, A}^{f_1, f_n}(k, \vec{x}, z) \approx \Ideal_{f, S}(k, \vec{x}, z)
\]
\end{defn}

\begin{thm}
If $f_1, \dots, f_m$ are secure protocol for computing $f_1, \dots, f_m$, and if $\Pi$ is a secure protocol for computing $f$ in the $(f_1, \dots, f_m)$-hybrid world, then the composed protocol $\Pi^{f_1, \dots, f_m}$ is a secure protocol for $f$.
\end{thm}

\subsubsection{Parallel Composition}

\putfigure{Parallel composition}{images/parallel_composition.png}
