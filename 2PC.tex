\section{Efficient and Maliciously Secure 2PC}

Ideas:
\begin{itemize}
\item use an OT protocol secure against malicious adversaries
\item $P_i$ can violate correctness by garbling the wrong circuit, or swapping OT inputs
\item the above can lead to violations of privacy
\item selective-failure attack on privacy
\end{itemize}

\subsection{Selective-Failure Attack}

\putfigure{Idea of preventing a selective-failure attack}{selective-failure-attack}

\begin{prcl}[Cut-and-Choose]
\putfigure{The cut-and-choose protocol to prevent selective-failure attacks}{cut-and-choose}
In this protocol, $P_1$ maximizes its probability of successfully cheating if $l/4$ circuits are bad.
So,
\begin{align*}
    \prob{\text{success}} 
    &=
    \frac{\binom{3 l / 4}{l / 2}}{\binom{l}{l / 2}}
    \\ &=
    \frac{(3l/4)! (l/2)!}{(l/4)! l!}
    \\ &=
    \frac{(l/2) \cdots (l/4 + 1)}{l \cdots (3l/4 + 1)}
    \\ &\leq 
    2^{-l/4}.
\end{align*}
Then the probability for cheating $2^{-n}$ implies $l \approx 4n$
(can be improved to $l \approx 3n$).
\end{prcl}

The previous protocol was the \textit{circuit-level} cut-and-choose protocol. The same idea can be achieved by using garbled gates at the \textit{gate-level}, as demonstrated in the following protocol.

\begin{prcl}[LEGO cut-and-choose]
This approach uses garbled circuits and applies the cut-and-choose idea to each gate of the garbled circuit.
\putfigure{The idea of the LEGO cut-and-choose protocol}{cut-and-choose-lego-idea}
For security $2^{-n}$, the total garbled gates needed is $O(|c| \cdot n / \log|c|$
TODO: more details at end of lecture 15
\end{prcl}